% On découpe ce document complexe en plusieurs sous-fichiers séparés.
% Cela permettra notamment de réarranger les transparents facilement 
% lors de l'élaboration du document.

% La définition de la classe beamer avec tous les styles afférents
\input{preambule/special_beamer.tex}

% Les autres packages utiles  notamment pour le français, les accents ou Python
\input{preambule/autres_packages.tex}
% Les macros et raccourcis personnels
\input{preambule/macros.tex}

% On définit le titre et l'auteur du document

% L'argument optionnel (entre crochets) donne le titre qui sera mis sur chaque slide
\title[GitHub en CPGE]{GitHub et GitHub classroom pour la gestion des TP en CPGE}
\author{Jean-Julien \textsc{Fleck}} % Votre nom
\institute{Lycée Kléber}
\date{Luminy 2019} 

% On démarre le document proprement dit
\begin{document}

% La page de titre et la table des matières
\input{slides/00_titre_et_tdm.tex}

% La première grande partie: introduction du sujet
% Titre de la premiere partie
\section{\texttt{git}, GitHub et GitHub Classroom}

%%%%%%%%%%%%%%%%%%%%%%%%%%%%%%%%%%%%%%%%%%%%%%%%
% Première diapo
%%%%%%%%%%%%%%%%%%%%%%%%%%%%%%%%%%%%%%%%%%%%%%%%
\begin{frame}
\frametitle{\texttt{git}}
\framesubtitle{Logiciel de gestion de versions décentralisé}

\begin{itemize}
	\item	 Logiciel libre créé par Linus Torvald en 2005
	\item  Permet des sauvegardes successives mais aussi le travail en parallèle de plusieurs personnes sur un même jeu de fichier (penser aux multiples versions des programmes de simulation en TIPE)
	\item  Accessible par la ligne de commande

	(pratique pour scripter mais pas trop pour les élèves)
	\item  Fonctionne en terme de \ofg{commit} qui rassemble des changements atomiques

\end{itemize}

\end{frame}


%%%%%%%%%%%%%%%%%%%%%%%%%%%%%%%%%%%%%%%%%%%%%%%%
% Deuxième diapo
%%%%%%%%%%%%%%%%%%%%%%%%%%%%%%%%%%%%%%%%%%%%%%%%
\begin{frame}
\frametitle{GitHub}
\framesubtitle{Une interface web pour \texttt{git}}

\begin{itemize}
	\item	<1->

\end{itemize}

\end{frame}


%%%%%%%%%%%%%%%%%%%%%%%%%%%%%%%%%%%%%%%%%%%%%%%%
% Troisième diapo
%%%%%%%%%%%%%%%%%%%%%%%%%%%%%%%%%%%%%%%%%%%%%%%%
\begin{frame}
\frametitle{GitHub Classroom}
\framesubtitle{Un service annexe de GitHub}

\begin{itemize}
	\item	<1->

\end{itemize}

\end{frame}


% La 2e partie: Mise en place de classroom
% Titre de la partie
\section{Mise en place de GitHub Classroom}


%%%%%%%%%%%%%%%%%%%%%%%%%%%%%%%%%%%%%%%%%%%%%%%%
%%%%%%%%%%%%%%%%%%%%%%%%%%%%%%%%%%%%%%%%%%%%%%%%
\begin{frame}
	\frametitle{Ensemble du processus}
	%\framesubtitle{}

	Il est décrit dans cette petite vidéo

	\begin{center}
		\url{https://www.youtube.com/watch?v=ChA_zph7aao}
	\end{center}


\end{frame}



%%%%%%%%%%%%%%%%%%%%%%%%%%%%%%%%%%%%%%%%%%%%%%%%
%%%%%%%%%%%%%%%%%%%%%%%%%%%%%%%%%%%%%%%%%%%%%%%%
\begin{frame}
	\frametitle{Création de l'organisation}
	%\framesubtitle{}

	\begin{center}
		\includegraphics[width=0.8\linewidth]{figures/classroom_organization.png}
	\end{center}

	\begin{itemize}[<+->]
		\item	Depuis \url{github.com}, aller dans «Settings» (en haut à droite sur l'avatar de votre compte)

		\item Puis «Organizations» (onglet en bas à gauche)

		\item Puis «New organisation» (bouton)

	\end{itemize}

\end{frame}


%%%%%%%%%%%%%%%%%%%%%%%%%%%%%%%%%%%%%%%%%%%%%%%%
%%%%%%%%%%%%%%%%%%%%%%%%%%%%%%%%%%%%%%%%%%%%%%%%
\begin{frame}
	\frametitle{Création de la \ofg{classroom}}
	\framesubtitle{Première étape}

	Depuis \url{https://classroom.github.com/classrooms} (après s'être connecté, bien entendu), cliquer sur le bouton «New classroom» (en haut à droite)

	\begin{center}
		\includegraphics[width=\linewidth]{figures/classroom_new_classroom.png}
	\end{center}

\end{frame}

%%%%%%%%%%%%%%%%%%%%%%%%%%%%%%%%%%%%%%%%%%%%%%%%
%%%%%%%%%%%%%%%%%%%%%%%%%%%%%%%%%%%%%%%%%%%%%%%%
\begin{frame}
	\frametitle{Création de la \ofg{classroom}}
	\framesubtitle{Sélection de l'organization}

	Sélectionner l'«organization» adéquate

	\begin{center}
		\includegraphics[width=\linewidth]{figures/classroom_select_organization.png}
	\end{center}

\end{frame}

%%%%%%%%%%%%%%%%%%%%%%%%%%%%%%%%%%%%%%%%%%%%%%%%
%%%%%%%%%%%%%%%%%%%%%%%%%%%%%%%%%%%%%%%%%%%%%%%%
\begin{frame}
	\frametitle{Création de la \ofg{classroom}}
	\framesubtitle{Nomination}

	Donner un nom explicite à la classroom

	\begin{center}
		\includegraphics[width=\linewidth]{figures/classroom_naming.png}
	\end{center}

\end{frame}

%%%%%%%%%%%%%%%%%%%%%%%%%%%%%%%%%%%%%%%%%%%%%%%%
% Deuxième diapo
%%%%%%%%%%%%%%%%%%%%%%%%%%%%%%%%%%%%%%%%%%%%%%%%
\begin{frame}
	\frametitle{Création de la \ofg{classroom}}
	\framesubtitle{Invitation d'autres administrateurs}

	%Inscrivez vos collègues à votre organization et donnez-leur le lien vers la classroom

	\begin{center}
		\includegraphics[width=\linewidth]{figures/classroom_invite_admin.png}
	\end{center}

\end{frame}

%%%%%%%%%%%%%%%%%%%%%%%%%%%%%%%%%%%%%%%%%%%%%%%%
% Deuxième diapo
%%%%%%%%%%%%%%%%%%%%%%%%%%%%%%%%%%%%%%%%%%%%%%%%
\begin{frame}
	\frametitle{Création de la \ofg{classroom}}
	\framesubtitle{Préparation du roster}

	%Inscrivez vos collègues à votre organization et donnez-leur le lien vers la classroom

	\begin{center}
		\includegraphics[width=\linewidth]{figures/classroom_roster.png}
	\end{center}

\end{frame}

%%%%%%%%%%%%%%%%%%%%%%%%%%%%%%%%%%%%%%%%%%%%%%%%
%%%%%%%%%%%%%%%%%%%%%%%%%%%%%%%%%%%%%%%%%%%%%%%%
\begin{frame}
	\frametitle{Premier «Assignment»}
	\framesubtitle{Création}

	%Inscrivez vos collègues à votre organization et donnez-leur le lien vers la classroom

	\begin{center}
		\includegraphics[width=\linewidth]{figures/classroom_assignment_creation.png}
	\end{center}

\end{frame}

%%%%%%%%%%%%%%%%%%%%%%%%%%%%%%%%%%%%%%%%%%%%%%%%
%%%%%%%%%%%%%%%%%%%%%%%%%%%%%%%%%%%%%%%%%%%%%%%%
\begin{frame}
	\frametitle{Premier «Assignment»}
	\framesubtitle{Type}

	%Inscrivez vos collègues à votre organization et donnez-leur le lien vers la classroom

	\begin{center}
		\includegraphics[width=\linewidth]{figures/classroom_assignment_type.png}
	\end{center}

\end{frame}

%%%%%%%%%%%%%%%%%%%%%%%%%%%%%%%%%%%%%%%%%%%%%%%%
%%%%%%%%%%%%%%%%%%%%%%%%%%%%%%%%%%%%%%%%%%%%%%%%
\begin{frame}
	\frametitle{Premier «Assignment»}
	\framesubtitle{Réglages}

	\begin{center}
		\includegraphics[height=0.8\textheight]{figures/classroom_assignment_reglages.png}
	\end{center}

\end{frame}

%%%%%%%%%%%%%%%%%%%%%%%%%%%%%%%%%%%%%%%%%%%%%%%%
%%%%%%%%%%%%%%%%%%%%%%%%%%%%%%%%%%%%%%%%%%%%%%%%
\begin{frame}
	\frametitle{Premier «Assignment»}
	\framesubtitle{Récupération des dossiers}

	\begin{center}
		\includegraphics[width=\linewidth]{figures/classroom_assignment_suivi.png}
	\end{center}

\end{frame}


% La 3e partie: Diverses utilisations possibles
\input{slides/03_troisieme_partie.tex}

% Conclusion
\input{slides/04_conclusion.tex}

\end{document}


