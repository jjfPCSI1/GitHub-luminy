% On découpe ce document complexe en plusieurs sous-fichiers séparés.
% Cela permettra notamment de réarranger les transparents facilement
% lors de l'élaboration du document.

% La définition de la classe beamer avec tous les styles afférents
%%%%%%%%%%%%%%%%%%%%%%%%%%%%%%%%%%%%%%%%%
% Beamer Presentation
% LaTeX Template
% Version 1.0 (10/11/12)
%
% This template has been downloaded from:
% http://www.LaTeXTemplates.com
%
% License:
% CC BY-NC-SA 3.0 (http://creativecommons.org/licenses/by-nc-sa/3.0/)
%
%%%%%%%%%%%%%%%%%%%%%%%%%%%%%%%%%%%%%%%%%

%----------------------------------------------------------------------------------------
%	PACKAGES AND THEMES
%----------------------------------------------------------------------------------------

\RequirePackage{currfile} 


\documentclass{beamer}


\mode<presentation> {

% The Beamer class comes with a number of default slide themes
% which change the colors and layouts of slides. Below this is a list
% of all the themes, uncomment each in turn to see what they look like.

%\usetheme{default}
%\usetheme{AnnArbor}
%\usetheme{Antibes}
%\usetheme{Bergen}
%\usetheme{Berkeley}
%\usetheme{Berlin}
%\usetheme{Boadilla}
%\usetheme{CambridgeUS}
%\usetheme{Copenhagen}
%\usetheme{Darmstadt}
%\usetheme{Dresden}
%\usetheme{Frankfurt}
%\usetheme{Goettingen}
%\usetheme{Hannover}
%\usetheme{Ilmenau}
%\usetheme{JuanLesPins}
%\usetheme{Luebeck}
%\usetheme{Madrid}		
%\usetheme{Malmoe}
%\usetheme{Marburg}
%\usetheme{Montpellier}
%\usetheme{PaloAlto}
%\usetheme{Pittsburgh}
%\usetheme{Rochester}
%\usetheme{Singapore}
%\usetheme{Szeged}
\usetheme{Warsaw}

% As well as themes, the Beamer class has a number of color themes
% for any slide theme. Uncomment each of these in turn to see how it
% changes the colors of your current slide theme.

%\usecolortheme{albatross}
%\usecolortheme{beaver}
%\usecolortheme{beetle}
%\usecolortheme{crane}
%\usecolortheme{dolphin}
%\usecolortheme{dove}
%\usecolortheme{fly}
%\usecolortheme{lily}
%\usecolortheme{orchid}
%\usecolortheme{rose}
%\usecolortheme{seagull}
%\usecolortheme{seahorse}
\usecolortheme{whale}
%\usecolortheme{wolverine}

%\setbeamertemplate{footline} % To remove the footer line in all slides uncomment this line
%\setbeamertemplate{footline}[frame number] % To replace the footer line in all slides with a simple slide count uncomment this line

%\setbeamertemplate{navigation symbols}{} % To remove the navigation symbols from the bottom of all slides uncomment this line

\setbeamercovered{transparent} % Fait apparaître les animations en grisé (utile pour la conception, mais peut être commenté lors de la remise du document final)

% Pour utiliser une police à empattements partout
\usefonttheme{serif}

% Pour rajouter la numérotation des frames dans les pieds de page
\newcommand*\oldmacro{}%
\let\oldmacro\insertshorttitle%
\renewcommand*\insertshorttitle{%
  \oldmacro\hfill%
  \insertframenumber\,/\,\inserttotalframenumber}

}

\usepackage{graphicx} % Allows including images
\usepackage{booktabs} % Allows the use of \toprule, \midrule and \bottomrule in tables




% Les autres packages utiles  notamment pour le français, les accents ou Python
\usepackage{natbib}         % Pour la bibliographie
\usepackage{url}            % Pour citer les adresses web
\usepackage[T1]{fontenc}    % Encodage des accents
\usepackage[utf8]{inputenc} % Lui aussi
\usepackage[frenchb]{babel} % Pour la traduction française
\usepackage{numprint}       % Histoire que les chiffres soient bien

\usepackage{amsmath}        % La base pour les maths
\usepackage{mathrsfs}       % Quelques symboles supplémentaires
\usepackage{amssymb}        % encore des symboles.
\usepackage{amsfonts}       % Des fontes, eg pour \mathbb.

\usepackage{cancel}

%\usepackage[svgnames]{xcolor} % De la couleur

%%% Si jamais vous voulez changer de police: décommentez les trois 
%\usepackage{tgpagella}
%\usepackage{tgadventor}
%\usepackage{inconsolata}

%%% Pour L'utilisation de Python
\usepackage{minted}
\usemintedstyle{friendly}

\usepackage{graphicx} % inclusion des graphiques
\usepackage{wrapfig}  % Dessins dans le texte.

\usepackage{tikz}     % Un package pour les dessins (utilisé pour l'environnement {code})
\usepackage[framemethod=TikZ]{mdframed}

% Les macros et raccourcis personnels
% Ce fichier contient toutes les macros que vous pouvez avoir envie de définir
% si vous les utilisez plusieurs fois dans le document.

\PassOptionsToPackage{svgnames}{color}

% Un environnement pour bien présenter le code informatique
\newenvironment{code}{%
\begin{mdframed}[linecolor=green,innerrightmargin=30pt,innerleftmargin=30pt,
backgroundcolor=black!5,
skipabove=10pt,skipbelow=10pt,roundcorner=5pt,
splitbottomskip=6pt,splittopskip=12pt]
}{%
\end{mdframed}
}

% Un raccourci pour composer les unités correctement (en droit)
% Exemple: $v = 10\U{m.s^{-1}}$
\newcommand{\U}[1]{~\mathrm{#1}}

% Les guillemets \ofg{par exemple}
\newcommand{\ofg}[1]{\og{}#1\fg{}}

% Le d des dérivées doit être droit: \frac{\dd x}{\dd t}
\newcommand{\dd}{\text{d}}

% La dérivée temporelle, tellement courante en physique, avec les d droits
\newcommand{\ddt}[1]{\frac{\dd #1}{\dd t}}

% Des parenthèses, crochets et accolades qui s'adaptent automatiquement à la
% taille de ce qu'il y a dedans
\newcommand{\pa}[1]{\left(#1\right)}
\newcommand{\pac}[1]{\left[#1\right]}
\newcommand{\paa}[1]{\left\{#1\right\}}

% Un raccourci pour écrire une constante
\newcommand{\cte}{\text{C}^{\text{te}}}

% Pour faire des indices en mode texte (comme les énergie potentielles)
\newcommand{\e}[1]{_{\text{#1}}}

% Le produit vectoriel a un nom bizarre:
\newcommand{\vectoriel}{\wedge}

\AtBeginSection[]{
  \begin{frame}
  \vfill
  \centering
  \begin{beamercolorbox}[sep=8pt,center,shadow=true,rounded=true]{title}
    \usebeamerfont{title}\insertsectionhead\par%
  \end{beamercolorbox}
  \vfill
  \end{frame}
}


% On définit le titre et l'auteur du document

% L'argument optionnel (entre crochets) donne le titre qui sera mis sur chaque slide
\title[GitHub en CPGE]{GitHub et GitHub classroom \\
pour la gestion des TP en CPGE}
\author{Jean-Julien \textsc{Fleck}} % Votre nom
\institute{Lycée Kléber}
\date{Luminy 2019}

% On démarre le document proprement dit
\begin{document}

% La page de titre et la table des matières
% Rien d'autre à faire qu'afficher le titre
\begin{frame}
\titlepage 
\end{frame}


% La table des matières utilise ce que vous donnez aux commandes \section et 
% \subsection tout au long de la présentation.
\begin{frame}
\frametitle{Plan de l'exposé} 
\tableofcontents 
\end{frame}


% La première grande partie: introduction du sujet
% Titre de la premiere partie
\section{\texttt{git}, GitHub et GitHub Classroom}

%%%%%%%%%%%%%%%%%%%%%%%%%%%%%%%%%%%%%%%%%%%%%%%%
% Première diapo
%%%%%%%%%%%%%%%%%%%%%%%%%%%%%%%%%%%%%%%%%%%%%%%%
\begin{frame}
\frametitle{\texttt{git}}
\framesubtitle{Logiciel de gestion de versions décentralisé}

\begin{itemize}
	\item	 Logiciel libre créé par Linus Torvald en 2005
	\item  Permet des sauvegardes successives mais aussi le travail en parallèle de plusieurs personnes sur un même jeu de fichier (penser aux multiples versions des programmes de simulation en TIPE)
	\item  Accessible par la ligne de commande

	(pratique pour scripter mais pas trop pour les élèves)
	\item  Fonctionne en terme de \ofg{commit} qui rassemble des changements atomiques

\end{itemize}

\end{frame}


%%%%%%%%%%%%%%%%%%%%%%%%%%%%%%%%%%%%%%%%%%%%%%%%
% Deuxième diapo
%%%%%%%%%%%%%%%%%%%%%%%%%%%%%%%%%%%%%%%%%%%%%%%%
\begin{frame}
\frametitle{GitHub}
\framesubtitle{Une interface web pour \texttt{git}}

\begin{itemize}
	\item	<1->

\end{itemize}

\end{frame}


%%%%%%%%%%%%%%%%%%%%%%%%%%%%%%%%%%%%%%%%%%%%%%%%
% Troisième diapo
%%%%%%%%%%%%%%%%%%%%%%%%%%%%%%%%%%%%%%%%%%%%%%%%
\begin{frame}
\frametitle{GitHub Classroom}
\framesubtitle{Un service annexe de GitHub}

\begin{itemize}
	\item	<1->

\end{itemize}

\end{frame}


% La 2e partie: Mise en place de classroom
% Titre de la partie
\section{Mise en place de GitHub Classroom}

%%%%%%%%%%%%%%%%%%%%%%%%%%%%%%%%%%%%%%%%%%%%%%%%
% Première diapo
%%%%%%%%%%%%%%%%%%%%%%%%%%%%%%%%%%%%%%%%%%%%%%%%
\begin{frame}
\frametitle{Création de l'organisation}
%\framesubtitle{}

\begin{itemize}
	\item	<1->

\end{itemize}

\end{frame}


%%%%%%%%%%%%%%%%%%%%%%%%%%%%%%%%%%%%%%%%%%%%%%%%
% Deuxième diapo
%%%%%%%%%%%%%%%%%%%%%%%%%%%%%%%%%%%%%%%%%%%%%%%%
\begin{frame}
\frametitle{Création de la \ofg{classroom}}
%\framesubtitle{}

\begin{itemize}
	\item	<1->

\end{itemize}

\end{frame}

%%%%%%%%%%%%%%%%%%%%%%%%%%%%%%%%%%%%%%%%%%%%%%%%
% Troisième diapo
%%%%%%%%%%%%%%%%%%%%%%%%%%%%%%%%%%%%%%%%%%%%%%%%
\begin{frame}
\frametitle{Préparation du roster}
%\framesubtitle{}

\begin{itemize}
	\item	<1->	
\end{itemize}

\end{frame}


% La 3e partie: Diverses utilisations possibles
% Le titre de la partie
\section{Diverses utilisation possibles}

%%%%%%%%%%%%%%%%%%%%%%%%%%%%%%%%%%%%%%%%%%%%%%%%
% Première diapo
%%%%%%%%%%%%%%%%%%%%%%%%%%%%%%%%%%%%%%%%%%%%%%%%

\begin{frame}
\frametitle{Un devoir par TP par élève}
%\framesubtitle{}

\begin{itemize}
	\item	<1->
\end{itemize}

\end{frame}

%%%%%%%%%%%%%%%%%%%%%%%%%%%%%%%%%%%%%%%%%%%%%%%%
% Deuxième diapo
%%%%%%%%%%%%%%%%%%%%%%%%%%%%%%%%%%%%%%%%%%%%%%%%

\begin{frame}
\frametitle{Un dossier semestriel par élève}
%\framesubtitle{}

\begin{itemize}
	\item	<1->
\end{itemize}

\end{frame}

%%%%%%%%%%%%%%%%%%%%%%%%%%%%%%%%%%%%%%%%%%%%%%%%
% Diapo exemple: le code informatique impose un
% environnement "fragile" pour la frame
%%%%%%%%%%%%%%%%%%%%%%%%%%%%%%%%%%%%%%%%%%%%%%%%

\begin{frame}[fragile]
\frametitle{Relativité Générale}
\framesubtitle{Le code informatique}

\begin{code}
\begin{minted}[linenos]{python}

import scipy as sp
import scipy.optimize

def ma_fonction(x):
    return []

# À vous de remplir les choses adéquates...

\end{minted}
\end{code}
\end{frame}


% Conclusion
% Le titre de la partie
\section{Conclusion}

%%%%%%%%%%%%%%%%%%%%%%%%%%%%%%%%%%%%%%%%%%%%%%%%
% Première diapo avec un exemple de tableau
%%%%%%%%%%%%%%%%%%%%%%%%%%%%%%%%%%%%%%%%%%%%%%%%
\begin{frame}
\frametitle{Conclusion}
\framesubtitle{sous forme de tableau}

\begin{table}
\begin{tabular}{c c c}
\toprule
\textbf{Newton} & \textbf{Rel. Restreinte} & \textbf{Rel. Générale}\\
\midrule
$531''/$siècle & ($+7''/$siècle) & $+43''/$siècle \\
\midrule
\multicolumn{3}{c}{Observations: $574''/$siècle} \\
\bottomrule
\end{tabular}
\caption{Effet des différentes théories}
\end{table}

\end{frame}



\end{document}
