% Titre de la premiere partie
\section{\texttt{git}, GitHub et GitHub Classroom}

%%%%%%%%%%%%%%%%%%%%%%%%%%%%%%%%%%%%%%%%%%%%%%%%
% Première diapo
%%%%%%%%%%%%%%%%%%%%%%%%%%%%%%%%%%%%%%%%%%%%%%%%
\begin{frame}
\frametitle{\texttt{git}}
\framesubtitle{Logiciel de gestion de versions décentralisé}

\begin{itemize}
	\item	 Logiciel libre créé par Linus Torvald en 2005
	\item  Permet des sauvegardes successives mais aussi le travail en parallèle de plusieurs personnes sur un même jeu de fichier (penser aux multiples versions des programmes de simulation en TIPE)
	\item  Accessible par la ligne de commande

	(pratique pour scripter mais pas trop pour les élèves)
	\item  Fonctionne en terme de \ofg{commit} qui rassemble des changements atomiques

\end{itemize}

\end{frame}


%%%%%%%%%%%%%%%%%%%%%%%%%%%%%%%%%%%%%%%%%%%%%%%%
% Deuxième diapo
%%%%%%%%%%%%%%%%%%%%%%%%%%%%%%%%%%%%%%%%%%%%%%%%
\begin{frame}
\frametitle{GitHub}
\framesubtitle{Une interface web pour \texttt{git}}

\begin{itemize}
	\item	<1->

\end{itemize}

\end{frame}


%%%%%%%%%%%%%%%%%%%%%%%%%%%%%%%%%%%%%%%%%%%%%%%%
% Troisième diapo
%%%%%%%%%%%%%%%%%%%%%%%%%%%%%%%%%%%%%%%%%%%%%%%%
\begin{frame}
\frametitle{GitHub Classroom}
\framesubtitle{Un service annexe de GitHub}

\begin{itemize}
	\item	<1->

\end{itemize}

\end{frame}
